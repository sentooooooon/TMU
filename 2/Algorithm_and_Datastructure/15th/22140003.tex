\documentclass[fleqn, a4paper. 12pt]{ltjsarticle} 
% following is the compile command (以下のコマンドでコンパイル可能)
% $ lualatex guidance.tex 
% you need to install texlive to execute the above.(texlive を入れる必要がある.)
\usepackage{amsmath,txfonts}
\usepackage{amssymb}
\usepackage{url}
\usepackage{subcaption}
\usepackage[margin=31mm]{geometry}
\usepackage{graphicx}
\usepackage{listings}


\begin{document}
\begin{titlepage}
      \begin{center}
      {
      \Huge 2023年度\\アルゴリズムとデータ構造 最終課題3}
      
      \vspace{4cm}
             {\Huge テクニカルライティング\\
               実験レポート\\}
             \vspace{4cm}
                    {\large 提出日:2024年2月6日(火)\\}
                    
                    {\large 学修番号:22140003\\氏名:佐倉仙汰郎}
    \end{center}  
  \end{titlepage}

  \section{コードと説明}
  \begin{lstlisting}[language=C++]
    void makeGraph(){
        ifstream file("graph_edgelist.txt");
        while(!file.eof()){
            int a,b;
            file >> a >>b;
            G[a][b] = 1;
            G[b][a] = 1;
        }
    }
    \end{lstlisting}
    ここではファイルから読み込み、隣接行列を作る.\\


    \begin{lstlisting}[language=C++]
        bool isConnected(){
    queue<int> q;
    int start = 0;
    q.push(start);

    while(!q.empty()){
        int v = q.front();
        visited[v] = true;
        q.pop();
        for(int i = 0; i < 34; i++){
            if(visited[i])continue;
            else if(G[v][i] == 1){
                q.push(i);
            }
        }
    }

    if(isvisited())return true;
    else{
        return false;
    }
    }
    \end{lstlisting}
    この関数で、連結行列の判定を行っている.BFSを使い到達可能なノードの数が、すべてのノードの数と等しいことを確認している.
    まずは、キューを作り今回は0をスタート地点として探索を開始する.
    while文を使いキューが空になるまで、探索をする.
    最初のノードに隣接しているノードをキューに追加し、最初のノードは探索済みとする.
    そして、キューの中にあるノードに隣接しているノードをキューに追加し、そのノードを探索済みとすることで、つながっているノードをすべて探索することができる.
    また注意点として、隣接しているノードをキューに追加するときに、探索済みのものはスキップしないと無限ループに入ってしまう.

    \section{連結かどうか?}
    連結である.
\end{document}