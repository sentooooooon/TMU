\documentclass[11pt, a4paper, titlepage]{jarticle}
\usepackage{geometry}
\usepackage{cite}
\usepackage{url}

\geometry{left=25mm,right=25mm,top=25mm,bottom=30mm}

\title{
2023年度 システムプログラミング実験\\
確率プログラミング\\
レポート}
\author{
学修番号: xxxxxxxx \\
氏名: aaa \\
}
\begin{document}
\date{
第1回レポート提出日:yyyy/mm/dd \\
第2回レポート提出日:yyyy/mm/dd \\
第3回レポート提出日:yyyy/mm/dd \\
}
\maketitle

\section*{はじめに}

\subsection*{実験の概要}

(本実験の概要を書く.)

\subsection*{実験環境}

(使用したプログラミング言語など,本実験に用いた環境を書く.)

\newpage
\section*{課題1-1}

(課題 1-1 のレポートを記載する.)

\newpage
\section*{課題x-y}

(以下同様に,各課題のレポートを記載する.
課題ごとに \textbackslash newpage を実行して,ページを切り替えること.)

\newpage
\section*{おわりに}

(本実験の感想など.)

\newpage
\bibliography{reference}
\bibliographystyle{junsrt} 

(参考文献.\cite{example} などを参考に,bib ファイルの使い方はウェブなどで勉強すること.)

\end{document}