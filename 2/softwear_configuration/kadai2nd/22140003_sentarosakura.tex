\documentclass[fleqn, a4paper. 12pt]{ltjsarticle} % lualatex guidance.tex
\usepackage{amsmath,txfonts}
\usepackage{amssymb}
\usepackage{url}
\usepackage[margin=31mm]{geometry}
\usepackage{graphicx}
\usepackage{color}
\usepackage{listings}
\usepackage{booktabs}
\usepackage{ascmac}
%ここからソースコードの表示に関する設定
\lstset{
  basicstyle={\linespread{0.6}},
  identifierstyle={\small},
  keywordstyle={\small\bfseries},
  ndkeywordstyle={\small},
  stringstyle={\small\ttfamily},
  frame={tb},
  breaklines=true,
  columns=[l]{fullflexible},
%  numbers=left,
  xrightmargin=1em,
  xleftmargin=2em,
  numberstyle={\scriptsize},
  stepnumber=1,
  numbersep=1em
}
\newcommand{\XA}[1]{\begingroup \color{red}}
\newcommand{\AX}[1]{\endgroup}
\lstnewenvironment{cppcode}
{\lstset{language=C++,
         basicstyle=\small\ttfamily,
         frame=single,
         breaklines=true,
         numbers=left}}
{}


\geometry{left=25mm,right=25mm,top=25mm,bottom=30mm}

\title{
2023年度 システム構成論レポート}
\author{
学修番号: 22140003 \\
氏名: 佐倉仙汰郎 \\
}
\begin{document}
\date{
レポート提出日:2024/1/19 
}
\maketitle

\newpage

\begin{itembox}[l]{課題1}
    授業で示した例以外で,基底クラス,派生クラスを自由に作成しなさい.また,作成したクラスに対して仮想関数を定義し,その機能を確認しなさい.
\end{itembox}


\begin{lstlisting}[caption=kadai1, label=program1]
    #include<iostream>
    #include<string>
    using namespace std;
    
    class character{
        private:
        string name_;
        public:
        character(string name) : name_(name) {}
    
        virtual void print_info() const = 0;
    };
    
    class Archer : public character{
        private:
        int hp = 100;
        int attack = 30;
        string job = "Archer";
        public:
        Archer(string name) : character(name) {}
        void print_info() const override{
            cout << "job :" << job << endl;
            cout << "hp :" << hp << endl;
            cout << "attack :" << attack << endl;
        }
    };
    
    class Warrior : public character{
        private:
        int hp = 250;
        int attack = 10;
        string job = "Warrior";
        public:
        Warrior(string name) : character(name) {}
        void print_info() const override{
            cout << "job :" << job << endl;
            cout << "hp :" << hp << endl;
            cout << "attack :" << attack << endl;
        }
    };
    
    
    
    
    int main(){
        Archer A("Satoshi");
        Warrior B("Yui");
    
        A.print_info();
        B.print_info();
        return 0;
    
    }
\end{lstlisting}

\noindent {\bf 出力例}
\begin{screen}
\begin{verbatim}
    job :Archer
    hp :100
    attack :30
    job :Warrior
    hp :250
    attack :10
\end{verbatim}
\end{screen}

\newpage

\begin{itembox}[l]{課題2}
    授業で示した複素数クラス COMPLEX の加算以外の四則演算 (減算,乗算,除算) を実行する演算子関数を作成しなさい.
    複素数 a, b に対して,a += b (= a + b) を実行する演算子関数 operator+=を作成しなさい.
\end{itembox}


\begin{lstlisting}[caption=kadai2, label=program2]
    #include<iostream>

    using namespace std;
    
    class COMPLEX
    {
    private:
    double re_;
    double im_;
    public:
    COMPLEX(double re = 0, double im = 0) : re_{re}, im_{im} {}
    double re() { return re_; }
    double im() { return im_; }
    friend COMPLEX operator+(const COMPLEX&, const COMPLEX&);
    friend COMPLEX operator-(const COMPLEX&, const COMPLEX&);
    friend COMPLEX operator*(const COMPLEX&, const COMPLEX&);
    friend COMPLEX operator/(const COMPLEX&, const COMPLEX&);
    COMPLEX& operator+=(const COMPLEX&);
    COMPLEX conjugate();
    friend ostream& operator<<(ostream &os, const COMPLEX& c);
    }; 
    
    
    
    COMPLEX operator-(const COMPLEX& X, const COMPLEX& Y){
        double r = X.re_ - Y.re_;
        double i = X.im_ - Y.im_;
        COMPLEX Z(r,i);
        return Z;
    }
    
    COMPLEX operator*(const COMPLEX& X, const COMPLEX& Y){
        double r = (X.re_*Y.re_) - (X.im_*Y.im_);
        double i = (X.im_*Y.re_) + (X.re_*Y.im_);
        COMPLEX Z(r,i);
        return Z;
    }
    
    COMPLEX operator/(const COMPLEX& X, const COMPLEX& Y){
        double r = ((X.re_*Y.re_) + (X.im_*Y.im_)) / ((Y.im_*Y.im_) + (Y.re_*Y.re_));
        double i = ((-X.re_*Y.im_) + (X.im_*Y.re_)) / ((Y.im_*Y.im_) + (Y.re_*Y.re_));
        COMPLEX Z(r,i);
        return Z;
    }
    
    
    int main(){
    
        COMPLEX x(3,10);
        COMPLEX y(1,1);
        //minus
        COMPLEX z = x - y;
        cout << z.re() << " + " <<  z.im() << "i" << endl;
    
        //multiple
        z = x * y;
        cout << z.re() << " + " <<  z.im() << "i" << endl;
    
        //devide
        z = x / y;
        cout << z.re() << " + " <<  z.im() << "i" << endl;
    
        return 0;
    }
\end{lstlisting}

\noindent {\bf 出力例}
\begin{screen}
\begin{verbatim}
    2 + 9i
    -7 + 13i
    6.5 + 3.5i
\end{verbatim}
\end{screen}

\newpage

\begin{itembox}[l]{課題3}
    次式で定義される関数 f(x) の値を返す関数オブジェクトを作成しなさい.
    以下の積分を台形近似により求める関数を作成し,$a = 1,x \rightarrow \infty$ に対して計算例を示しなさい (フレネル積分).

\end{itembox}


\begin{lstlisting}[caption=kadai3, label=program3]
    #include<iostream>
    #include<cmath>
    
    using namespace std;
    
    class f{
        private:
        double a_;
        public:
        f(double p) : a_{p} {}
        ~f() {}
    
        double operator()(const double &x) const {
            return cos(a_ * x * x);
        }
    };
    
    double func(double x){
        double a = 1.0;
        return cos(a * x * x);
    }
    
    double fresnel(double (*F)(double)){
        double sum = 0;
        double gosa = 0.01;
        double d = 0.01;
        double x0 = 0;
        int count = 0;
        while(true){
            double X = (F(x0) + F(x0 + d));
            x0 += d;
            sum += X;
            count ++;
            if(count == 1000)break;
            cout << X << endl;
        }
        return sum;
    }
    
    int main(){
        cout << fresnel(func) << endl;
        return 0;
    }
\end{lstlisting}

\noindent {\bf 出力例}
\begin{screen}
\begin{verbatim}
    none
\end{verbatim}
\end{screen}

\newpage

\begin{itembox}[l]{課題4}
    整数,実数,string 型等,大小関係が定義されている任意の型 T のオブジェクト a,bを入力とする.a と b のうち大きい方を返す関数を関数テンプレートにより作成し,その動作を確認しなさい.また,この関数テンプレートが動作するクラスを 1 つ定義し,その動作を確認しなさい.
\end{itembox}


\begin{lstlisting}[caption=kadai4, label=program4]
    #include <iostream>
    #include <string>
    
    template <typename T>
    T max_value(const T& a, const T& b) {
        return (a > b) ? a : b;
    }
    
    template <typename T>
    class ExampleClass {
    public:
        ExampleClass(const T& val) : value(val) {}
    
        T getValue() const {
            return value;
        }
    
        ExampleClass<T> max(const ExampleClass<T>& other) const {
            return ExampleClass<T>(max_value(value, other.getValue()));
        }
    
    private:
        T value;
    };
    
    int main() {
        int int_a = 5, int_b = 8;
        std::cout << max_value(int_a, int_b) << std::endl;
    
        double double_a = 3.14, double_b = 2.71;
        std::cout << max_value(double_a, double_b) << std::endl;
    
        std::string str_a = "apple", str_b = "banana";
        std::cout << max_value(str_a, str_b) << std::endl;
    
        ExampleClass<int> int_instance_a(10), int_instance_b(15);
        ExampleClass<int> max_int_instance = int_instance_a.max(int_instance_b);
        std::cout << max_int_instance.getValue() << std::endl;
    
        ExampleClass<std::string> str_instance_a("cat"), str_instance_b("dog");
        ExampleClass<std::string> max_str_instance = str_instance_a.max(str_instance_b);
        std::cout << max_str_instance.getValue() << std::endl;
    
        return 0;
    }
    
\end{lstlisting}

\noindent {\bf 出力例}
\begin{screen}
\begin{verbatim}
    Max of integers: 8
    Max of doubles: 3.14
    Max of strings: banana
    Max of ExampleClass<int>: 15
    Max of ExampleClass<std::string>: dog
\end{verbatim}
\end{screen}

\end{document}