\documentclass[fleqn, a4paper. 12pt]{ltjsarticle} % lualatex guidance.tex
\usepackage{amsmath,txfonts}
\usepackage{amssymb}
\usepackage{url}
\usepackage[margin=31mm]{geometry}
\usepackage{graphicx}
\usepackage{color}
\usepackage{listings}
\usepackage{booktabs}
\usepackage{verbatim}
%ここからソースコードの表示に関する設定
\lstset{
  basicstyle={\linespread{0.6}},
  identifierstyle={\small},
  keywordstyle={\small\bfseries},
  ndkeywordstyle={\small},
  stringstyle={\small\ttfamily},
  frame={tb},
  breaklines=true,
  columns=[l]{fullflexible},
%  numbers=left,
  xrightmargin=1em,
  xleftmargin=2em,
  numberstyle={\scriptsize},
  stepnumber=1,
  numbersep=1em
}
\newcommand{\XA}[1]{\begingroup \color{red}}
\newcommand{\AX}[1]{\endgroup}
\lstnewenvironment{cppcode}
{\lstset{language=C++,
         basicstyle=\small\ttfamily,
         frame=single,
         breaklines=true,
         numbers=left}}
{}


\geometry{left=25mm,right=25mm,top=25mm,bottom=30mm}

\title{
2023年度 ソフトウェア構成論\\
レポート1}
\author{
学修番号: 22140003 \\
氏名: 佐倉仙汰郎 \\
}
\begin{document}
\date{
第1回レポート提出日:2023/12/22 
}
\maketitle

\newpage

\section{問題1}
問題 1:円の面積を求める関数をマクロ置換により作成し,半径 30 の円の面積を求めなさい.ただし,円周率は C 言語の標準ライブラリより読み込むこと.

\begin{lstlisting}[language = C++]
    #include<iostream>
#include<cmath>

using namespace std;

#define PI 3.1414
#define area(r) (M_PI * r * r)

int main(){
    double r = 30.0;
    double area = area(r);

    cout << area <<endl;
    return 0;
}    
\end{lstlisting}
出力結果
\verbatiminput{kadai1.txt}

\section{問題2}

\begin{lstlisting}[language = C++]
    #include<iostream>
#include<cmath>

using namespace std;

void calc(double *xn){
    *xn = sqrt(*xn) + 1;
}

int main(){
    double x0;
    for(int i =0; i < 20; i++){
        calc(&x0);
        cout << x0 << endl;
    }

    return 0;
}   
\end{lstlisting}
出力結果
\verbatiminput{kadai2.txt}

\section{問題3}

\begin{lstlisting}[language = C++]
    /*問題 3:stdlib.h の qsort 関数など,標準ライブラリに含まれているソート関数では,
2 つの変数を比較するための関数を定義することにより,並べ替えの基準を自由に設定
することができる.独自の比較関数を定義し,並べ替えを実行した例を示しなさい.並
べ替える変数は構造体など何でもよい.*/

#include <stdlib.h>
#include<iostream>

using namespace std;

int comp( const void* lhs, const void* rhs ) {
    return *(char*)lhs - *(char*)rhs;
}

typedef struct {
    char s;
} test;

int main() {
    // 構造体を要素とする配列を作成
    test array[] = {
        {'e'},
        {'a'},
        {'k'},
        {'y'},
        {'u'}
    };

    size_t array_size = sizeof(array) / sizeof(array[0]);

    qsort(array, array_size, sizeof(test), comp);

    for (size_t i = 0; i < array_size; ++i) {
        printf("%c ", array[i].s);
    }
    return 0;
}
\end{lstlisting}
出力結果
\verbatiminput{kadai3.txt}

\section{問題4}

\begin{lstlisting}[language = C++]
    #include<iostream>

using namespace std;

void plus5_with_pointer(int* x){
    *x = *x + 5;
}

void plus5_without_pointer(int x){
    x = x + 5;
}

int main(){
    int x = 5;
    int *p = &x;

    cout << "ポインタを使ってアドレスを使う方法はいくつかある.以下では2つのやり方を示す." << endl;
    cout << "xのアドレス: "<< p << endl;
    cout << "xのアドレス: "<< &x << endl;
    cout << "値を表す方法も示す." << endl;
    cout << "xの値: " << *p << endl;

    cout << "xの値を関数を用いて変更したいときのもポインタが使える." << endl;
    plus5_without_pointer(x);
    cout << "ポインタを使わない関数で+5した場合、x: " << x << endl;
    plus5_with_pointer(p);
    cout << "ポインタを使った関数で+5したとき、x: " << x << endl;

    cout << "----------------------------------------" << endl;
    cout << "ポインタは配列についても有用である." << endl;
    int vals[] = { 1 ,1 ,2 ,3 ,5, 8, 13};
    int *valptr = vals;
    cout << "配列の先頭の値: "<<*valptr << endl;
    for(int i = 0; i < 7; i++){
        cout  << *(valptr + i) << " ";
    }
    return 0;


    return 0;
}
\end{lstlisting}
出力結果
\verbatiminput{kadai4.txt}
\end{document}